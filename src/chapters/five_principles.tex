\chapter{The Five Principles}
\epigraph{Within humanity is goodness that is alive and fully intact but, in these times, it is surrounded by the darkness of uncertainty and fear.}{\textit{Sakyong Mipham, The Shambhala Principle: Discovering Humanity's Hidden Treasure}}

According to Buddhist wisdom, the source of all suffering is our attachment to ultimately transitory phenomena. 
We constantly grasp after things we want, push away things we dislike, and ignore things we don't really
care about. As we continue down this path, we become more and more self-involved, unable to see that,
ultimately, we have capabilities which, when put to full use, can give us a deep well-spring of contentment
and joy. When we can tap into our basically good nature, we are able to see things as they are,
and with that wisdom, have a greater impact on humankind.

The principles we will be talking about are grounded in the intersection of two streams: Buddhist and
Agile. The Buddhist one, over 2,500 years old, was kicked off by Siddharta Gautama when he realized
a ``middle way'' for finding personal awakening. You can find more about Buddhism all over the place.
A good place to start might be \url{http://www.buddhanet.net/}.

The Agile one, initiated in 2001, is much more recent,
initiated by a group of consultants and practitioners interested in reawakening the human side of
software development. For more on that, visit \url{http://www.agilemanifesto.org}.

In each section below, we will cover five principles: \textbf{Rest}, \textbf{Rhythm}, \textbf{Relationship}, 
\textbf{Results}, and \textbf{Renewal}. All five have been infused with core Buddhist wisdom. Each of them,
if you look hard enough, can be found in the values and principles in the above-mentioned Agile Manifesto. 
For each principle, I will share my view on its deeper meaning and what it means in the context of working 
with your mind as well as your team.

\subsection{Warning}

The five principles you are about to explore will help you create a deeper sense of self-awareness. Once you
go down this road, there is no turning back. You can't ``un-see'' your own mind. In order to become more
capable of working with the suffering you live with every day, you have to be able to look at it. With this
awareness comes greater intensity. It may become unbearable at times. Just remember: you have always
had this suffering --- you were simply unaware of it.

This new level of sensitivity that you will cultivate will open you up. Being more available to others, you
will suddenly find that they are naturally drawn to you without knowing why. Something subtle happens when
you are more inquisitive. You're touching into a wellspring of wisdom that naturally invites the best in them.

On the flip side, you may find that the sudden shift in your ability to perceive is overwhelming. This is why
we begin with the principle of \textbf{Rest}. There is nothing wrong with finding your limit, backing off a bit, 
and gathering up the will to take the next leap forward. 

You are not the first to have walked this path by a long shot. As you continue to explore, you might find
it helpful to find someone who has gone down this path before you. Find a Buddhist meditation center near
you and inquire about instruction and/or mentorship. They will be more than happy to help!

I personally recommend the Shambhala Buddhist path. There are hundreds of centers scattered all over
the world. Visit \url{http://www.shambhala.org} to learn more.

\chapter{Rest}


\section{Rhythm}


\chapter{Relationship}


\section{Results}


\section{Renewal}

