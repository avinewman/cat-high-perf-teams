\chapter{Controversial Bits}
\epigraph{A leader is best when people barely know he exists, when his work is done, his aim fulfilled, they will say: we did it ourselves.}{\textit{Lao Tzu}}

Why do so many business books quote Lao Tzu? Or Confucius? These are prominent figures in Taoism
and Confucianism, which are Eastern religions.

Well, one big reason is that, in Eastern cultures, it is difficult to separate where ``business'' or ``government''
begins and where spirituality ends. That is to say, there is no real distinction. They are all part of one whole 
human experience. Only here in the West have we created separate domains, including the spiritual, which 
are considered distinct things.

It's safe to say that, in this book, there will be a potentially uncomfortable blurring of the lines between business
and spirituality. My main hypothesis for why this is necessary: business can be enriched by integrating spiritual
ideas. Why? When we have meaningful interactions with coworkers, we create stronger connections with them.
And with those connections come a higher likelihood of helping and being helped. When we are all helping
one another, the work we do is better, faster, and more valuable.

This particular book is heavily inspired by Shambhala Buddhism, a religion that arose out of Tibetan Buddhism,
and propagated by Ch�gyam Trungpa Rimpoche. He was the first to present the four dignities that we will be
introducing and form the ground upon which we stand. These dignities encompass, in only a few 
pithy words, all of Buddhist ethics, practice, and psychology. It is from this core that we will be drawing
much of our inspiration.

As well, there is a heavy emphasis on meditation practice. For some, this simple yet powerful approach to 
working with the mind verges on magical. However, as you'll hopefully see as you read on, nothing
could be further from the truth. This is a practice, powerful approach to becoming familiar with your own
mind, which allows you to move more and more into the driver's seat, instead of being ruled by unconscious
habit and emotion.