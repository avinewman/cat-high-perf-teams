\chapter{Introduction}

My path has been a fairly twisty one. From high school, I was always interested in computers, as well as
spirituality. I didn't know it at the time, but those two thought systems would remain fairly separate in my mind
for over a decade, only to merge into a single whole. In hindsight, though, it might have been easy to see. As I 
was graduating from college and bidding goodbye to my therapist, he said something that will stick with me 
forever. ``I believe you will find a way to merge technology and spirituality in a way nobody ever predicted.'' 
Perhaps this is the beginning of that endeavor!

The question of how to make a high-performing team has been on our collective minds for quite some time.
As we have evolved, becoming capable of seeing greater and greater levels of complexity, we invented
more, subtler methods for establishing, controlling, cajoling, incentivizing, and so on, all in the name of progress.

My first experience with this approach was my first job out of college. I had absolutely no idea what ``good''
was for a software team, but from the beginning I was dissatisfied. It seemed like people were wasting time on 
things that could have been more effectively done if we could only figure out how to automate more manual
repetitive tasks, and do them more often. That way, we would always stay ``close to the ground'' as we were
doing our work. We were successful. Tasks that were taking hours a week were reduced to minutes, and 
we used our highly-valued programming skills to do more creative things.

As I matured, I began to see that, while useful, removing time waste was only one method for helping
teams of software developers become more capable. During my time learning about human systems, 
psychology, and even Buddhism, the real problem finally started coming into focus. All the automation in
the world wasn't going to help if a team never formed into a cohesive unit. If team members didn't have the
skills to share ideas and adapt themselves, they were nothing better than a group of individuals. 

With social capacity, however, greater things are possible. New ideas form that no one person could have
ever come up with. Through the creative evolutionary process, teams could discover amazing new ways
to produce benefit for the world.

\section{History}

Here in the West we have been studying the human mind for a strikingly short amount of time. Our
level of maturity could be considered almost primitive in comparison to our Eastern brethren. The
individualism that has permeated our culture has created a sense that progress is the most important
thing we can be striving for. We are constantly looking outside of ourselves for answers to life's 
great questions. As we do this, we reinforce habits of looking to others when things go wrong. We
have a need for some kind of savior, since we are clearly incapable of handling ourselves.

In contrast, Taoist, Buddhist, and other Eastern cultures have turned inward. Their experiments have
to do with the very stuff of which thoughts are made. Breaking down every mental phenomenon, their 
ability to analyze, reason about, and ultimately transform the mind is far ahead of our own. These
collectivist cultures, while advancing technologically much more slowly, have learned how to work 
together quite well. It's no surprise --- when you are able to work with your own mind, you 
are better able to understand how others' minds will work.

\section{Exploration}

The exploration we will be embarking on in this book is how we take the idea of social capacity 
further. What is possible if everyone on a team invests in teamwork skills? When they are in a mature, almost 
enlightened state, what can they create together? What kind of transformative ideas could emerge from 
such a team?

The questions we ask are almost as important as the methods we invoke to get there. They can serve
to narrow or focus or broaden it. They can take us outside ourselves, calling forth something greater, or
they can keep us locked in a cycle of fear and doubt. We will be focusing on broadening and calling
forth that higher vision.

\section{Intention}

With this understanding, we begin our journey. You will learn five major principles that can help
you begin to work with your mind, understand how collective efforts work, and ultimately become
a more effective team member. As you grow in skill, your teammates will start to notice. They will 
suddenly want to be around you more, as you are able to call forth the best in them.

Let's get started! 